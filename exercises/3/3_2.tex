\documentclass[12pt,letterpaper,boxed]{hmcpset}
\usepackage{amsthm}
\usepackage{amsmath}
\usepackage[margin=1in]{geometry}
\usepackage{graphicx}

\newtheorem{definition}{Definition}[section]

\newcommand{\f}[2]{\textnormal{\textbf{#1}}\ #2\ }
\newcommand{\val}[1]{\textnormal{\textbf{#1}}}

% info for header block in upper right hand corner
\name{}
\class{}
\assignment{}
\duedate{}

\begin{document}

\problemlist{}

\begin{definition}[3.2.3]
    For each natural number $i$, define a set $S_i$ as follows:
    \begin{align*}
        S_0 &= \emptyset \\
        S_{i+1} &= \{\val{true}, \val{false}, \val{0}\} \\
                &\cup \{\f{succ}{t_1}, \f{pred}{t_1}, \f{iszero}{t_1} \mid t_1 \in S_i \} \\
                &\cup \{\f{if}{t_1} \f{then}{t_2} \f{else}{t_3} \mid t_1, t_2, t_3 \in S_i\}
    \end{align*}
    \begin{align*}
        S = \bigcup_i S_i
    \end{align*}
\end{definition}

\begin{problem}[3.2.4]
How many elements does $S_3$ have?
\end{problem}

\begin{solution}
    \begin{align*}
        |S_0| &= 0 \\
        |S_1| &= 3 + 0 + 0 = 3 \\
        |S_2| &= 3 + 3*3 + 3^3 = 39 \\
        |S_3| &= 3 + 39*3 + 39^3 = 59,439 
    \end{align*}
\end{solution}

\begin{problem}[3.2.5]
Show that the sets $S_i$ are \emph{cumulative} -- that is, for each $i$ we have $S_i \subseteq S_{i+1}$.
\end{problem}

\begin{solution}
    
\end{solution}

\end{document}
